\chapter{Fingerprints}
\label{chapter:fingerprints}

This chapter introduces fingerprints---the representation used by our approach
for security-sensitive operations. As outlined in \chapref{chapter:overview},
fingerprints are matched against source code to locate security-sensitive
operations and are thus central to our approach.

A server receives and processes client requests to access and modify the
resources that it manages. Each such client request may perform one or more
security-sensitive operations on the resource, each of which must be mediated
by an authorization policy lookup. For example, an \xclient's request to map
the child windows of a window \code{pParent} using a call to
\code{MapSubWindows} (see \figref{figure:mapsubwindows}) results in the
security-sensitive operations of enumerating the children of \code{pParent}
(denoted by \op{Window\_Enumerate}) and that of mapping each of the child
windows onto the screen (denoted by \op{Window\_Map}). Each such
security-sensitive operation can have one or more \textit{fingerprints}, 
where each fingerprint denotes the resource accesses needed to perform that
security-sensitive operation.


%------------------------------------------------------------------------------
\section{Syntax}
\label{chapter:fingerprints:syntax}

The syntax of a fingerprint of a security-sensitive operation \op{OP} is as
defined in \figref{figure:fingerprint-definition}. A fingerprint is a rule,
where the left-hand-side of the rule is the name of a security-sensitive
operation, \op{OP}, and the right hand side of the rule is a conjunction
of one or more \textit{code patterns} (also called \textit{code templates})
or their negations.

%------------------------------------------------------------------------------
\begin{figure}[ht!]
\begin{center}
\begin{tabular}{|r c l|}
%
\hline
\bnfnterm{Fingerprint} 
  & ::== & 
  \op{OP} :- ($\displaystyle\bigwedge_{i=1}^{n}$$_{Intra}$ \bnfnterm{CP$_i$})
  \textit{Subject to} \bnfnterm{Conditions}\\
%
  & & $|$ 
  \op{OP} :- ($\displaystyle\bigwedge_{i=1}^{n}$$_{Inter}$ \bnfnterm{CP$_i$})
  \textit{Subject to} \bnfnterm{Conditions}\\
%  
\bnfnterm{CP} 
  & ::== & 
  \bnfnterm{Code-Pattern}  $|$ $\neg$\bnfnterm{Code-Pattern}\\
%
\bnfnterm{Condition}
  & ::== & 
  \textit{Same}(\bnfnterm{Code-Pattern}$_i$, \bnfnterm{Code-Pattern}$_j$)\\
% 
  & & $|$
  \textit{Different}(\bnfnterm{Code-Pattern}$_i$, \bnfnterm{Code-Pattern}$_j$)\\
% 
\bnfnterm{Conditions}
  & ::== & 
  \bnfnterm{Condition} $\wedge$ \bnfnterm{Conditions} $|$ True\\
%
\bnfnterm{Code-Pattern} 
  & ::== &
  \textit{Write} \bnfnterm{Value} \textit{To} \bnfnterm{AST}\\
%
  & & $|$
  \textit{Read} \bnfnterm{AST}\\
%
  & & $|$
  \textit{Call} \bnfnterm{AST}\\
%
  & & $|$
  \textit{CallWith} \bnfnterm{AST} (\bnfnterm{Value}, \bnfnterm{Value}, $\ldots$)\\ 
%
  & & $|$
  \bnfnterm{BinaryRelation} (\bnfnterm{AST}, \bnfnterm{AST})\\
%
  & & $|$
  \bnfnterm{BinaryRelation} (\bnfnterm{AST}, \bnfnterm{Value})\\
%
  & & $|$
  \bnfnterm{UnaryRelation} (\bnfnterm{AST})\\
%
\bnfnterm{AST} 
  & ::== &
  (\textit{type-name}\texttt{->})$^\texttt{+}$\textit{fieldname}$^\texttt{*}$\\
%
\bnfnterm{Value}
  & ::== &
  \unk (unknown) $|$ \textit{constant}\\ 
%
  & & $|$
  \textit{Binary Arithmetic Operator} (\bnfnterm{Value}, \bnfnterm{Value})\\
%
\bnfnterm{BinaryRelation} & ::== & $\neq$ $|$ $==$\\
%
\bnfnterm{UnaryRelation} & ::== & \textit{Decrement} $|$ \textit{Increment}\\
\hline
\end{tabular}
\end{center}
\mycaption{BNF grammar defining fingerprints.  The symbol \op{OP} in the
definition of \bnfnterm{Fingerprint} denotes the name of a security-sensitive
operation. Note that an abstract syntax tree (AST) representing a resource
access is represented using data types, denoting the data type of the resource
being accessed.}
{\label{figure:fingerprint-definition}}
\end{figure}
%------------------------------------------------------------------------------

A code pattern is a \textit{Read}, \textit{Write} or \textit{Call} operation on
a resource; the resource is represented in the code pattern as an abstract
syntax tree (AST). In this dissertation, we restrict ourselves to the analysis
of C source code, and assume that resources are represented as C structures
(\code{struct}s). The ASTs in code patterns thus represent accesses to fields
of data types representing resources.  A few simple extensions of the above
operations are also included in the definition of a code pattern,
\eg~\textit{CallWith}, which denotes a \textit{Call} with constraints on actual
parameters, and simple binary and unary relations, such as tests for equality
and disequality, and the increment and decrement operator.

We make the following observations on the syntax of fingerprints.

\begin{enumerate}

\item 
Resources are expressed in fingerprints using data types, rather than
individual variable names. This decision was motivated by two considerations.

First, specifying resources using data types makes a fingerprint easy to write.
A security-sensitive operation can simply be expressed in terms of the data
types of the resources that it manipulates. Each piece of code that matches
this fingerprint (as described in \chapref{chapter:matching}) performs this
fingerprint. Second, using data types for resources relieves the matching
algorithm from having to use precise alias information. The matching algorithm
instead abstracts each variable in the program to its data type before checking
for a match. The matching algorithm is thus conservative---it may return
spurious matches (\ie~false positives), but will never miss a piece of code
that matches a fingerprint (\ie~false negatives). A false negative means that
an authorization check is not placed where it should be, thus resulting in a
potentially exploitable security hole.

\item 
Even though resources are expressed in terms of data types, we have found that
in some cases further constraints on code patterns in a fingerprint improves
the precision of matching. The syntax of fingerprints allows \textit{Same} and
\textit{Different} constraints that can be used to restrict the code fragments
that match a fingerprint. For example, in the fingerprint for
\op{Window\_Enumerate} shown below, the \textit{Different} constraint mentions
that the variable of type \op{WindowPtr} that matches the first code pattern
must be different from the variable that matches the second code pattern.

\begin{center}
\begin{tabular}{|r l|}
\hline
\op{Window\_Enumerate} :- 
  & \textit{Read}~\code{WindowPtr$_1$->firstChild}\\
	& $\wedge$ \textit{Read}~\code{WindowPtr$_2$->nextSib}\\
	& $\wedge$ \code{WindowPtr $\neq$ 0} \textit{Subject to}\\
	& \textit{Different}(\code{WindowPtr$_1$}, \code{WindowPtr$_2$})\\\hline
\end{tabular}
\end{center}

Note, however, that to avoid false negatives, precise alias information must be
used if such constraints are used in fingerprints. For example,
\textit{Same}(\code{WindowPtr$_1$}, \code{WindowPtr$_2$}) is a constraint that
restricts \code{WindowPtr$_1$} and \code{WindowPtr$_2$} to be the same,
\ie~point to the same resource. Alias information is needed to resolve this
constraint precisely. 

\item
Fingerprints can either be specified as \textit{intraprocedural} or
\textit{interprocedural} (represented in \figref{figure:fingerprint-definition}
using $\wedge_{Intra}$ and $\wedge_{Inter}$, respectively). The matching
algorithm described in \chapref{chapter:matching} matches intraprocedural
fingerprints against code contained in a procedure, while it matches
interprocedural fingerprints against code contained in all procedures in the
program being analyzed. In most cases in our experiments, however, we found
that intraprocedural fingerprints were sufficient to represent
security-sensitive operations.

\item 
Finally, we note that temporal ordering information cannot be expressed using
the syntax of fingerprints shown in \figref{figure:fingerprint-definition}.
Thus, we cannot express fingerprints to represent the rule ``Read
\code{WindowPtr->firstChild} before reading \code{WindowPtr->nextSib}''. While
the simpler fingerprint language results in simpler, more intuitive fingerprint
mining algorithms, the inability to express temporal ordering information is a
limitation, which can potentially result in false positives in the output of
the matching algorithm.

However, in our experiments (described in \chapref{chapter:matching}), we found
that the number of false positives was manageable. For example, in our analysis
of the \xserver, we found that one source of false positives was the
\op{Window\_Enumerate} fingerprint. The fingerprint for this security-sensitive
operation only approximates linked-list traversal, and thus triggers spurious
matches. In particular, out of $20$ locations that were output by the matching
algorithm as performing \op{Window\_Enumerate}, only $10$ did.

Extending the fingerprint language to include temporal information is a topic
for future work. Existing tools, such as MOPS~\cite{cw02} and
\code{xgcc}~\cite{hcx+02}, already employ algorithms to match such temporal
patterns and our work can benefit directly from these tools. Note, however,
that mining temporal fingerprints requires designing more sophisticated
algorithms than those developed in this dissertation. 

\end{enumerate}


%------------------------------------------------------------------------------
\section{Interpretation}
\label{chapter:fingerprints:meaning}

The fingerprint of a security-sensitive operation \op{OP} represents the
resource accesses needed to perform that operation. Thus, the effect that
a security-sensitive operation has on a resource is defined by the resource
accesses in its fingerprint.

Using the above interpretation, however, the fingerprint of a
security-sensitive operation \op{OP} must contain all the resource accesses
needed to perform that operation on the resource. For example, one fingerprint
for the operation \op{Window\_Map} of the \xserver, which represents mapping a
window onto the screen, is \textit{Call}~\code{MapWindow}, or alternatively,
the set of statements that implements \code{MapWindow}.  Expressing
fingerprints with \textit{all} the resource accesses needed to perform an
operation is certainly acceptable, but makes the fingerprint ineffective in the
presence of even minor changes to the source code of the server. 

As a result, we typically express fingerprints using only the set of resource
accesses that suffice to differentiate one security-sensitive operation from
another. For example, we found that the following fingerprint suffices to
precisely identify all locations in the \xserver\ that perform
\op{Window\_Map}:

\begin{center}
\begin{tabular}{|r l|}
\hline
\op{Window\_Map} :-
& \textit{Write} \code{True} \textit{To} \code{WindowPtr->mapped} $\wedge$ \\
& \textit{Write} \code{MapNotify} \textit{To} \code{xEvent->union->type}\\
\hline
\end{tabular}
\end{center}

The problem of expressing fingerprints at an appropriate granularity that
precisely represents security-sensitive operations is referred to by Erlingsson
as \textit{security event synthesis}~\cite[Pages 73--82]{e04}. This is the
subject of the next two chapters, which develop \textit{mining} algorithms to
extract fingerprints by analyzing source code.

