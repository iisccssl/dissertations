\chapter{Introduction}
\label{chapter:introduction}

% Short summary of this dissertation

This dissertation presents program analysis and transformation techniques to
retrofit legacy software with mechanisms for authorization policy enforcement.
Using case studies with real-world software systems, it demonstrates that these
techniques offer a principled and automated alternative to the \adhoc\ and
manual techniques that are currently used to retrofit legacy software.


%==============================================================================
\section{Motivation}
\label{chapter:introduction:motivation}

% Motivation for this thesis, the need for retroactive techniques to add

Design for Security, the principle that security must be a key design
consideration in the construction of a secure system, has long been a mantra of
the security community. Indeed, systems such as Multics~\cite{cv65} and
Hydra~\cite{wcc+74}, which provided strong security guarantees, were
constructed with security as a principle design consideration. 

While proactively designing for security will undoubtedly create more robust
and secure systems, doing so is often difficult in practice because of two
reasons.

First, because application functionality and performance are often more
tangible features to a customer, software producers focus on these aspects,
with security typically being an afterthought. Indeed, excluding the operating
system, the number of applications that have been proactively designed for
security is far outnumbered by applications that are not. Some examples of
software that have proactively been designed for security include the Postfix
mail program~\cite{postfix} and database servers~\cite{oracle,sqlserver}.
There are thus large bodies of legacy software with inadequate or non-existent
security mechanisms. It is impractical to require that these systems be
redesigned and rebuilt for security. It is instead advisable to
\textit{retrofit} security mechanisms into these systems. However, because of
the time and cost involved, even these security retrofits are currently
undertaken only rarely for large legacy systems.  For example, it took almost
two years to modify the Linux kernel to add mechanisms that enforce mandatory
access control (MAC) policies~\footnote{The Orange Book~\cite{tcsec} defines
Mandatory access control as ``a means of restricting access to objects based on
the sensitivity (as represented by a security label) of the information
contained in the objects and the formal authorization (\ie~clearance) of
subjects to access information of such sensitivity''. In contrast,
Discretionary access control (DAC) is defined as ``a means of restricting
access to objects based on the identity of subjects and/or groups to which they
belong. The controls are discretionary in the sense that a subject with a
certain access permission is capable of passing that permission (perhaps
indirectly) on to any other subject''. Thus MAC denies users full control over
access to resources, including those that they create. In contrast, in DAC,
ownership of a resource allows a subject full control over access to the
resource, including delegation of access rights to the resource.  Examples of
MAC policies include the Bell-LaPadula policy~\cite{bl76} and the Biba
policy~\cite{b77}, while the Graham-Denning model~\cite{gd72} is an example of
a DAC policy. The specifics of MAC and DAC are not central to the techniques
developed in this dissertation and will not be presented in further detail.
For a good overview of MAC and DAC, see McLean~\cite{m90}.}.
Similarly, the privilege-separated version of OpenSSH required a new system
architecture, and the addition of a significant amount of new
code~\cite{pfh03}.

Second, even if a system is designed for security, additional security
mechanisms may have to be added in the future. For example, while the Linux
kernel has historically had mechanisms to enforce discretionary access control
(DAC) policies, mechanisms to enforce MAC policies were added only in
2002~\cite{wcs+02}. Moreover, prior experience shows that even if a system is
designed for security, functional and performance enhancements that are added
in the future may break security assumptions, thus calling for a reevaluation
of the system's security. For example, Karger and Schell~\cite{ks74} showed
that modifications to improve the usability of Multics broke several of its
design assumptions, which resulted in potentially exploitable vulnerabilities.

These reasons motivate the need for retroactive techniques to secure
software---automatic and semi-automatic techniques to reason about security
properties of legacy software, and retrofit it with security mechanisms.
Indeed, the need for such techniques has also been realized by others. For
example, the CCured tool~\cite{nch+05,nmw02} automatically retrofits legacy
C~programs into a type-safe variant by inserting runtime checks that enforce
type safety. Similarly, the Privtrans tool~\cite{bs04} semi-automatically
refactors legacy C~programs for privilege separation using program
partitioning.

This dissertation continues this line of research into retroactive techniques
to secure legacy software, and focuses on the problem of retrofitting legacy
software with mechanisms for authorization policy enforcement.


%==============================================================================
\section{Retrofitting authorization policy enforcement mechanisms}
\label{chapter:introduction:retrofitting}

Software systems that manage shared resources must protect these resources from
unauthorized access. This is achieved by formulating and enforcing an
appropriate authorization policy (also called an access control policy).  This
policy specifies the set of \textit{security-sensitive operations} that a
\textit{subject} (typically a user of the system) can perform on an
\textit{object} (typically a resource managed by the system).  For example,
operating systems manage shared resources such as files, network connections
and memory, and typically enforce policies that determine how users of the
system can access these resources. A popular example of such a policy on
UNIX-like systems allows only the \code{root} user to perform the operation
\op{Write} on the \code{/etc/passwd} file (the resource).

Authorization policies are typically enforced using a security mechanism called
a \textit{Reference Monitor}. Introduced by Anderson in 1972~\cite{a72}, a
reference monitor is an entity that satisfies three key properties.

\begin{enumerate}

\item \textbf{Complete mediation.} The reference monitor must be invoked at
each access to a shared resource, \ie~all security-sensitive operations
performed on a shared resource must be mediated. Saltzer and Schroeder also
call this property the \textit{Principle of Complete Mediation}~\cite{ss75}.

\item \textbf{Tamper resistance.} The reference monitor mechanism must
be tamperproof, \ie~an attacker must not be able to circumvent the mechanism,
\eg~by rewriting the code of the reference monitor, so that an access check is
not performed (and the authorization policy not enforced) before a shared
resource is accessed. 

\item \textbf{Verifiability.} The reference monitor must be a small-enough
entity so as to allow for thorough verification.

\end{enumerate}

A reference monitor thus mediates each security-sensitive operation on a shared
resource, and ensures that a subject is allowed to perform the operation on a
resource only if it is allowed by the authorization policy. This dissertation
develops a suite of techniques to retrofit a reference monitor into a legacy
software system that lacks this security mechanism. 

We consider two concrete examples to motivate the need to retrofit a reference
monitor into legacy code: (1) the integration of the Linux Security Modules
(LSM) framework~\cite{wcs+02} to the Linux operating system, and (2) the
integration of a reference monitor with the \xserver~\cite{xserver}.

While Linux has long had the ability to enforce DAC policies, the need was felt
to extend these mechanisms to enforce MAC policies. This was achieved using the
Linux security modules (LSM) framework~\cite{wcs+02}, by retrofitting the Linux
kernel. The creation of the LSM framework was motivated by the proliferation of
different Linux patches that aimed to improve the default Linux access control
mechanism through finer grained enforcement of MAC
policies~\cite{argus,grsecurity,rsbac,nsa,apparmor,lids}. The LSM framework
generalizes this work to define a reference monitor interface for mediating all
accesses to security-sensitive operations via loadable kernel modules in a
policy-independent way. While there were differences between these patches in
scope, policy models, and ancillary features, a single reference monitor
interface can be defined that subsumes all approaches since the goal of a
reference monitor is complete mediation of all security-sensitive operations.
The development of the LSM framework from the multitude of Linux patches
described above was a manual process of collecting authorization decision
points (\ie~calls to the reference monitor, or \textit{hooks}) from each patch
and resolving inconsistencies between these choices. 

The \xserver, like many other server applications, enables multiple \xclients\
to access shared resources (\eg~windows, fonts) that it manages. However, the
\xserver\ was historically developed to promote cooperation between \xclients,
and security (\eg~isolation) of \xclients\ was not built into the design of the
server. In the absence of isolation within the \xserver, malicious clients can
compromise the integrity and privacy of other \xclients\ handled by the
\xserver---well-documented instances of such attacks abound in the literature
(\eg~\cite{ep91,ksv03,k04,w96a}). For example, in the \xserver, a malicious
\xclient\ can easily compromise the privacy of other \xclients\ by snooping on
their input, or by retrieving bitmaps of their windows~\cite{ksv03}. Similarly,
it is easy to program a Trojan horse \xclient\ that registers with the
\xserver\ to receive keystrokes sent to other \xclients\ connected to the
\xserver~(\eg~the xkey application~\cite{xkey} does so in just $100$
lines of C source code).

As was the case with Linux, several security mechanisms have been developed to
secure the \xserver\ (\eg~the X security extension~\cite{w96b}, and several
solutions at the level of the X protocol~\cite{dru05,w96a}).  The goal of the
X11/\selinux\ project~\cite{ksv03,w07} is to add a reference monitor interface
to the \xserver\ that can be used to enforce policies on how \xclients\ access
resources managed by the \xserver. This reference monitor is designed to
interface to the security-enhanced Linux (\selinux) policy server~\cite{ksv03}.

The central theme in both cases is to \textit{retrofit legacy software with a
reference monitor}. Other recent efforts with similar objectives include
retrofitting the Java Virtual Machine~\cite{f06}, the IBM Websphere
software~\cite{hms06,s07} and IBM DB2~\cite{s07} with mechanisms to enforce
\selinux\ authorization policies.  In addition to these examples, a tremendous
amount of other legacy code exists that likely requires similar retrofitting,
ranging from server applications (\eg~middleware~\cite{jdbc,odbc}, web
servers~\cite{apache,iis}, Samba~\cite{samba}, game systems~\cite{games}, proxy
and cache servers~\cite{squid}), to client applications that manage multiple
information flows (\eg~email clients~\cite{ham06}, browsers and chat servers).  


%==============================================================================
\section{Current practice}
\label{chapter:introduction:state-of-the-art}

In current practice, legacy software is manually retrofitted with a reference
monitor. We illustrate the steps involved in this process using the example in
\figref{figure:introduction:motivating-example}. 

This example shows a fragment of code from an API function \textsf{RequestAPI}
of a hypothetical server application. This API function accepts two arguments,
a \code{client} and a \code{target}, and performs security-sensitive operations
on an internal resource \code{obj}, that is derived from \code{target}. This is
reminiscent of, for instance, a user requesting a security-sensitive operation
(\eg~\op{Read}, \op{Write}, \op{Append}) on a file via a system call---much as
the system call internally translates the file into an inode before performing
the operation on the actual data blocks representing the file, this API
function translates \code{target} into \code{obj} by calling the function
\textsf{GetData}. The lines with underlined comments must be retrofitted to
ensure that security-sensitive operations are mediated by reference monitor
calls.  Retrofitting proceeds in four steps, as discussed below.


%%%%%%%%%%%%%%%%%%%%%%%%%%%%%%%%%%%%%%%%%%%%%%%%%%%%%%%%%%%%%%%%%%%%%%%%%%%%%%%
\begin{figure*}[ht!]
%
\begin{center}
\newsavebox{\motivatingexample}
\begin{lrbox}{\motivatingexample}
\begin{minipage}[ht]{6.0in}
\lstset{
language=C,
tabsize=4,
basicstyle=\ttfamily\scriptsize,
keywordstyle=\sffamily\bfseries,
commentstyle=\rmfamily\emph\scriptsize,
morekeywords={RequestAPI, GetSubjLabel, GetObjLabel, AuthHook, CheckPolicy},
morecomment=[s][\rmfamily\bfseries\emph\scriptsize\underbar]{/**}{*/},
escapeinside={/*@}{@*/}
}
\begin{lstlisting}[numbers=left, firstnumber=101]
/* API function to process a request by client on target */
void RequestAPI (client, target) {
    struct object obj; /*@ \hfill @*/ /* Denotes the resource to protect accesses to */
    Others x, y, z, subjlabel; /*@ \hfill @*/ /* Variable declarations */
    ...
    subjlabel = GetSubjLabel(client); /*@ \hfill @*/ /** Find subject label for client */ /*@ \label{line:motivating-example:getsublabel} @*/
    obj = GetData(target); /*@ \hfill @*/ /* Get obj from target */
    obj.label = GetObjLabel(target); /*@ \hfill @*/ /** Find object label for target */ /*@ \label{line:motivating-example:getobjlabel} @*/
    x = obj.innocuous; /*@ \hfill @*/ /* Non-security-sensitive operation on obj */ /*@ \label{line:motivating-example:innocuous} @*/
    if (AuthHook(subjlabel, obj.label)) { /*@ \hfill @*/ /** Reference monitor call */ /*@ \label{line:motivating-example:authhook} @*/
        y = obj.secret; /*@ \hfill @*/ /* Security-sensitive operation to read secret data */ /*@ \label{line:motivating-example:secrecy} @*/
        obj.integrity = z; /*@ \hfill @*/ /* Security-sensitive operation to write data */ /*@ \label{line:motivating-example:integrity} @*/
    }
    return;
}
\end{lstlisting}

\begin{lstlisting}[numbers=left, firstnumber=201]
/* Two reference monitor calls (CheckPolicy) in one authorization query */
int AuthHook (sublabel, objlabel) {
    QueryResults q1, q2; /*@ \hfill @*/ /* Variable declarations */

    q1 = CheckPolicy(sublabel, objlabel, SecrecyOp); /*@ \hfill @*/ /** Query for line /*@\ref{line:motivating-example:secrecy}@*/ */ /*@ \label{line:motivating-example:secrecychk} @*/
    q2 = CheckPolicy(sublabel, objlabel, IntegrityOp); /*@ \hfill @*/ /** Query for line /*@\ref{line:motivating-example:integrity}@*/ */ /*@ \label{line:motivating-example:integritychk} @*/
    return q1 && q2; 
}
\end{lstlisting}
\end{minipage}
\end{lrbox}\fbox{\usebox{\motivatingexample}}
\end{center}
%
\mycaption{Retrofitting legacy software with a reference monitor interface. 
Lines with underlined comments must be retrofitted.}
{\label{figure:introduction:motivating-example}}
\end{figure*}
%%%%%%%%%%%%%%%%%%%%%%%%%%%%%%%%%%%%%%%%%%%%%%%%%%%%%%%%%%%%%%%%%%%%%%%%%%%%%%%

\begin{enumerate}

\item \textbf{Subject/Object identification and labeling.} Each subject
requesting a security-sensitive operation and each object that may be affected
by the security-sensitive operation must be identified and labeled with a
security identifier.  Subjects and objects are represented in the authorization
policy using their labels, and the reference monitor uses these labels to
determine whether a requested operation is permitted. 

In current practice, the subjects and objects that are affected by a
security-sensitive operation are determined manually. Their labels are
bootstrapped using operating system support and are typically bound to the
variables representing the subject and object, \eg~stored as a field in the C
\code{struct} representing the subject and object
variables~\cite{ksv03,hms06,hrj+07,f06}.  For example, the \selinux\ operating
system~\cite{ls01a,m04} maintains security labels for each user and resource
(\eg~files, sockets) that it manages. Thus, the functions shown in
lines~\ref{line:motivating-example:getsublabel}
and~\ref{line:motivating-example:getobjlabel}, that fetch the subject and
object label, respectively, rely on the operating system to supply labels.  In
the example in \figref{figure:introduction:motivating-example}, the object
label is stored in the field \code{label} of \code{struct object}.  Note that
the application itself may create new objects, in which case it must also label
these objects appropriately. In current practice, the function calls to
determine subject and object labels
(lines~\ref{line:motivating-example:getsublabel}
and~\ref{line:motivating-example:getobjlabel}) are placed manually.


\item \textbf{Identifying security-sensitive operations.}
Lines~\ref{line:motivating-example:innocuous},
\ref{line:motivating-example:secrecy}
and~\ref{line:motivating-example:integrity} show accesses to members of the
variable \code{obj} of type \code{struct object}. In this example, \code{struct
object} denotes the type of a resource that is managed by the application. One
or more of these accesses may represent a \textit{security-sensitive
operation}. Intuitively, a security-sensitive operation is a conceptual
operation on a resource, such as a combination of structure-member accesses
(\eg~a set of structure-member accesses), that achieves a high-level objective.
For example, a combination of accesses to the inode structure in the Linux
kernel that performs a file read will be classified as a security-sensitive
operation.

Whether a combination of structure member accesses indeed represents a
security-sensitive operation or not is determined by site-specific security
requirements. For this example, we assume that the structure member accesses in
lines~\ref{line:motivating-example:secrecy}
and~\ref{line:motivating-example:integrity} represent distinct
security-sensitive operations (namely \op{SecrecyOp} and \op{IntegrityOp}),
while the structure member access in
line~\ref{line:motivating-example:innocuous} is not representative of any
security-sensitive operation.

In current practice, security-sensitive operations are determined manually.
Typically, a team of security analysts reasons about the kinds of security
policies that must be enforced by the software, and determines a set of
resources and security-sensitive operations.  For instance, the LSM project
identified $504$ distinct security-sensitive operations (such as
\op{File\_Read}, \op{File\_Write}, \op{Dir\_Mkdir}, \op{Dir\_Rmdir}) on
different resources, such as files, directories, sockets and shared memory,
that the Linux kernel (version $2.4.21$) manages~\cite{ls01a,svs01,s05b}.
Similarly, the X11/\selinux\ project identified $59$ distinct
security-sensitive operations (such as \op{Window\_Create}, \op{Window\_Map})
on resources such as windows, fonts and other resources that the \xserver\
manages~\cite{ksv03}.

It is important to note that security-sensitive operations are currently
identified using \adhoc\ reasoning, \eg~by considering different security
policies to be enforced, and \textit{not} by analyzing source code. As a result,
the relationship between the security-sensitive operations so identified and
the code that implements them is \textit{not} identified (\ie~the combination
of structure member accesses that represents a security-sensitive operation is
not identified). Thus, identifying where a security-sensitive operation happens
in source code is currently a manual and \adhoc\ process.


\item \textbf{Placing authorization queries.} Because the structure member
access on line~\ref{line:motivating-example:secrecy} represents the
security-sensitive operation \op{SecrecyOp}, it must be mediated by the
reference monitor call shown on line~\ref{line:motivating-example:secrecychk}
(in this case, the keyword \code{SecrecyOp} shown on
line~\ref{line:motivating-example:secrecychk} is a constant that denotes a
security-sensitive operation). Similarly, the structure member access on
line~\ref{line:motivating-example:integrity} must be mediated by the reference
monitor call on line~\ref{line:motivating-example:integritychk}.

Both these objectives are achieved by calling the function \textsf{AuthHook} on
line~\ref{line:motivating-example:authhook} with the subject \code{client} and
object \code{obj} that are involved in the security-sensitive operation. The
implementation of \textsf{AuthHook} is provided as part of the retrofitting
process.  In this case, the function \textsf{AuthHook} consults the
authorization policy (via the \textsf{CheckPolicy} function calls on
lines~\ref{line:motivating-example:secrecychk}
and~\ref{line:motivating-example:integritychk}) to check that both the
security-sensitive operations \op{SecrecyOp} and \op{IntegrityOp} are allowed.
They are thus either performed together, or are not performed at all.
Finer-grained placement of authorization checks, that will allow individual
checking of permissions for each of these security-sensitive operations is also
possible, \eg~by placing the function call on
line~\ref{line:motivating-example:secrecychk} guarding
line~\ref{line:motivating-example:secrecy}, and the function call on
line~\ref{line:motivating-example:integritychk} guarding
line~\ref{line:motivating-example:integrity}, respectively. In this case, the
programmer likely combined the checks for both these security-sensitive
operations because of one of three reasons:

\begin{itemize}
%
\item The site-specific policy demands that both \op{SecrecyOp} and
\op{IntegrityOp} be performed together, or not at all. In this case, it can be
argued that these security-sensitive operations instead be combined into a
single security-sensitive operation, called \op{SecrecyAndIntegrityOp}, that is
identified in code by a read of the \code{secret} field and a write of the
\code{integrity} field of a variable of type \code{struct object}. The
site-specific policy must also be rewritten to express policies for this
security-sensitive operation.
%
\item \op{SecrecyOp} and \op{IntegrityOp} are performed together at several
locations in code, and the programmer combined the checks for these operations
as an optimization (at the cost of a conservative authorization check).
%
\item This was unintended, and the programmer instead meant to check for
\op{SecrecyOp} and \op{IntegrityOp} separately.
%
\end{itemize}

In current practice, locating security-sensitive operations in source code, and
placing authorization queries to mediate them is a manual and \adhoc\ process.


\item \textbf{Writing an authorization policy.} The final component of the
retrofitting process is to write an appropriate authorization policy that
satisfies site-specific security goals. Abstractly, an authorization policy is
a set of triples
\triple{\subjlab{Subject-label}}{\subjlab{Object-label}}{\op{Operation}} that
determines the set of security-sensitive operations that a subject with label
\subjlab{Subject-label} can perform on on an object with label
\subjlab{Object-label}. The \textsf{CheckPolicy} function calls on
lines~\ref{line:motivating-example:secrecychk}
and~\ref{line:motivating-example:integritychk} consult the authorization policy
and determine whether the requested operation should be permitted (and return a
non-zero value if permitted). 

Because policies are determined by site-specific security goals, in current
practice, they are written and maintained manually. Several off-the-shelf
tools, such as the Tresys \selinux\ policy management
toolkit~\cite{tresys1,tresys2}, are now available to manage and interface with
authorization policies.

\end{enumerate}

As this example illustrates, retrofitting legacy code with a reference monitor
is currently an \adhoc, manual process. Thus, it is both
\textit{time-consuming} and \textit{error-prone}. For example, in spite of all
the prior work put into several security patches for the Linux
kernel~\cite{argus,grsecurity,rsbac,nsa,apparmor,lids}, it still took over two
years to integrate the LSM framework into the mainline Linux kernel, and a
number of bugs (\eg~showing violation of the Principle of Complete
Mediation~\cite{ss75}) were found and fixed along the way~\cite{jez04,zej02}.
Similarly, despite an initial implementation in 2003, the work of integrating a
reference monitor interface into the \xserver\ is still ongoing. Part of this
is due to changes in developers and the challenges of implementing a trusted
path for the user~(as outlined in several prior papers on secure windowing
systems~\cite{bpw+90,e90,emo+93,ep91,mpr06}), but recent work is still
addressing fundamental issues, such as subject/object labeling and design of
the query interface to the reference monitor~\cite{w07}.

              
%==============================================================================
\section{Contributions}
\label{chapter:introduction:contributions}

This dissertation develops techniques to automate the hitherto \adhoc, manual
process of retrofitting legacy software with a reference monitor.  The thesis
that this dissertation supports is the following:

\begin{center}
\newsavebox{\thesisstatement}
\begin{lrbox}{\thesisstatement}
\begin{minipage}[ht]{6.0in}
\textbf{Program analysis and transformation techniques offer a
principled and automated way to retrofit legacy software with mechanisms 
for authorization policy enforcement}
\end{minipage}
\end{lrbox}\fbox{\usebox{\thesisstatement}}
\end{center}

This dissertation supports the above thesis by making the following
contributions:

\begin{enumerate}

\item \textbf{Fingerprints.} It introduces a new formalism, called
\textit{fingerprints}, to represent security-sensitive operations.  A
fingerprint represents a security-sensitive operation using a set of
\textit{code patterns} that represent how a resource must be accessed to
perform that operation. Thus, a fingerprint implicitly embodies the
relationship between a security-sensitive operation and the code that embodies
this operation.

\item \textbf{Fingerprint mining algorithms.} It presents static and dynamic
program analysis algorithms to automatically mine fingerprints for
security-sensitive operations. The dynamic program analysis algorithm uses a
novel \textit{trace localization technique} that uses side-effects to localize
fingerprints in program traces. The static program analysis algorithm makes
novel use of a hierarchical clustering technique called \textit{concept
analysis} to mine fingerprints.  

Fingerprint mining algorithms remove the manual burden associated with Step~(2)
of the retrofitting process, discussed in
\sectref{chapter:introduction:state-of-the-art}.

\item \textbf{Fingerprint matching algorithms.} It presents a
\textit{fingerprint matching algorithm} that statically matches the fingerprint
of a security-sensitive operation against source code and identifies all
locations where the operation is performed. In conjunction with a program
transformation tool that places authorization queries, this algorithm
automatically retrofits legacy software with a reference monitor. It also
presents heuristics to identify the subject and object involved in a
security-sensitive operation. 

Fingerprint matching algorithms thus ameliorate the manual burden associated
with Steps~(1) and~(3) of the retrofitting process from
\sectref{chapter:introduction:state-of-the-art}.

\item \textbf{Case studies on real-word software.} This dissertation presents
case studies on three real-world software systems---the \ext\ file system from
Linux, the \xserver, and the PennMUSH multi-user dungeon~\cite{games}. The case
studies on \ext\ and \xserver\ directly compare the results of applying the
algorithms developed in this dissertation against the results obtained by
manually retrofitting these systems (as was done in the LSM and the
X11/\selinux\ projects, respectively), while the case study on PennMUSH
presents the applicability of these algorithms on a system that has as yet not
been manually retrofitted with a reference monitor.

\end{enumerate}

Techniques to help with Step~(4) from
\sectref{chapter:introduction:state-of-the-art}, namely, writing authorization
policies, are a subject of several past and ongoing research projects, and not 
considered in this dissertation. 


%==============================================================================
\section{Structure of this dissertation} 
\label{chapter:introduction:structure}
%
This dissertation is organized as follows.  \chapref{chapter:overview} presents
a high-level overview of our approach and states the assumptions underlying our
work.  \chapref{chapter:fingerprints} introduces fingerprints.
\chapref{chapter:dynamic} presents an algorithm to mine fingerprints from
runtime execution traces of a program and shows its application to the
\xserver.  \chapref{chapter:static} presents a static program analysis-based
fingerprint mining algorithm that overcomes several shortcomings of the dynamic
program analysis-based algorithm. It also presents the application of this
algorithm to the \ext\ file system, the \xserver\ and PennMUSH.
\chapref{chapter:matching} presents an algorithm to match fingerprints against
source code and place reference monitor checks.  \chapref{chapter:relatedwork}
presents related research and \chapref{chapter:conclusions} concludes with
directions for future work.
%

%==============================================================================
\section{Bibliographic attributions}
\label{chapter:introduction:biblio}

Most of the material presented in this dissertation has appeared in conference
papers.

\begin{itemize}

\item Parts of \chapref{chapter:fingerprints} and \chapref{chapter:matching}
appeared in Proceedings of the 12$^{th}$ ACM Conference on Computer and
Communications Security (Alexandria, Virginia, November 2005)~\cite{gjj05} as
joint work with T.~Jaeger and S.~Jha. This paper introduced fingerprints, and
presented an algorithm to match fingerprints against source code.

\item Parts of \chapref{chapter:overview}, \chapref{chapter:dynamic} and
\chapref{chapter:matching} appeared in Proceedings of the 27$^{th}$ IEEE
Symposium on Security and Privacy (Berkeley/Oakland, California, May
2006)~\cite{gjj06} as joint work with T.~Jaeger and S.~Jha. This paper
presented a dynamic fingerprint mining algorithm and applied it to the
\xserver.

\item Parts of \chapref{chapter:static} appeared in Proceedings of the
27$^{th}$ International Conference on Software Engineering (Minneapolis,
Minnesota, May 2007)~\cite{gkj+07} as joint work with D.~King, T.~Jaeger and
S.~Jha. This paper presented a static fingerprint mining algorithm using
concept analysis and applied it to the \ext\ file system, the \xserver, and
PennMUSH.

\end{itemize}

