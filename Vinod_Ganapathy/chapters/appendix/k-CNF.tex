\chapter{\textsf{\kcnf\ SET EQUATION} is NP-Complete}
\label{chapter:appendix:k-cnf}

We show that the \kcnfse\ problem that was introduced in
\chapref{chapter:dynamic} is NP-complete. 

\begin{definition}[\textsf{CNF set equation}]
%
A CNF set equation of a set $B$ in terms of a collection of sets $\mathcal{C}$
is an equation for $B$ in terms of the sets in $\mathcal{C}$ with $\cup$ as the
disjunction operator, and $\cap$ as the conjunction operator. Each disjunct in
a CNF set equation is called a \textit{clause}.
%
\end{definition}


\begin{definition}[The \kcnfse\ problem]
%
\textbf{INSTANCE:} Given (i)~a set $S$, (ii)~a set $B~\subseteq~S$, 
(iii)~a collection $\mathcal{C}$ of subsets of $S$, and (iv)~a positive
integer $k$.
%
\textbf{QUESTION:} Is there a CNF set-equation for $B$ in terms of the
sets in $\mathcal{C}$ with at most $k$ clauses?
%
\end{definition}

We proceed by reduction from the \setcov\ problem, which is known to be
NP-complete~\cite{garey-johnson}.


\begin{definition}[The \setcov\ problem]
%
\textbf{INSTANCE:} Given (i)~a set $S$, (ii)~a collection $\mathcal{C}$ of
subsets of $S$, and (iii)~a positive integer $k$.
%
\textbf{QUESTION:} Is there a collection of at most $k$ sets from $\mathcal{C}$ 
that covers $S$?
%
\end{definition}



\noindent\textbf{Reduction from \setcov:} Choose $k$ elements
\{$1, 2, \ldots, k$\} that do not belong to the set $S$. Construct
$B~=~\{1, 2, \ldots, k$\}, and a collection $\mathcal{D}$ from $\mathcal{C}$ as
follows. Let $\mathcal{C}$~=~\{$C_1, C_2, \ldots, C_n$\}.  Construct
$\mathcal{D}$~=~\{\{$1$\}$\cup$$\bar{C_1}$, \{$1$\}$\cup$$\bar{C_2}$, $\ldots$,
\{$1$\}$\cup$$\bar{C_n}$, \{$2$\}$\cup$$\bar{C_1}$, \{$2$\}$\cup$$\bar{C_2}$,
$\ldots$ \{$2$\}$\cup$$\bar{C_n}$, $\ldots$, \{$k$\}$\cup$$\bar{C_1}$,
\{$k$\}$\cup$$\bar{C_2}$, $\ldots$, \{$k$\}$\cup$$\bar{C_n}$\}. Here the
complements are constructed with $S$ as the universal set. Pass $B$,
$\mathcal{D}$, and $k$ to the oracle that solves \kcnfse\ and ouput ``Yes'' for
\setcov\ if and only if the oracle \kcnfse\ also returns ``Yes''.  Clearly,
this reduction is polynomial-time. 

\begin{clm}
$S$ has a set cover of size $k$ if and only if $B$ has a \kcnf\ set equation in
terms of $\mathcal{D}$.
\end{clm}

\noindent\textbf{Proof:}
($\Rightarrow$) Suppose that $S$ has a set cover of size $k$. Let the cover be
$S~=~A_1\cup A_2\cup\ldots\cup A_k$ (these sets are elements of the collection
$\mathcal{C}$). Note that this implies that
$\phi=\bar{A_1}\cap\bar{A_2}\cap\ldots\cap\bar{A_k}$.
Consequently, a \kcnf\ set equation for $B$ is $\mathcal{B}$~=~
((\{$1$\}$\cup$$\bar{A_1}$) $\cup$ (\{$2$\}$\cup$$\bar{A_1}$) $\cup$ $\ldots$
$\cup$ (\{$k$\}$\cup$$\bar{A_1}$))
$\cap$
((\{$1$\}$\cup$$\bar{A_2}$) $\cup$ (\{$2$\}$\cup$$\bar{A_2}$) $\cup$ $\ldots$
$\cup$ (\{$k$\}$\cup$$\bar{A_2}$))
$\cap$$\ldots$$\cap$
((\{$1$\}$\cup$$\bar{A_k}$) $\cup$ (\{$2$\}$\cup$$\bar{A_k}$) $\cup$ $\ldots$
$\cup$ (\{$k$\}$\cup$$\bar{A_k}$)).

($\Leftarrow$) Suppose that $B$ has a \kcnf\ set equation. Denote this set
equation by $B~=~B_1\cap B_2\cap\ldots\cap B_k$, where each $B_i$ is a
disjunction of elements from $\mathcal{D}$. Note that $B$ is a strict subset of
each of $B_i$ (because the elements $1, 2, \ldots, k$ do not belong to $S$).
Consider the sets $G_1, \ldots, G_k$, defined as $G_i = B_i - B$. Each of these
sets $G_i$ is a finite union of complements of sets from $\mathcal{C}$ (because
of the way $\mathcal{D}$ is constructed). Further
$\displaystyle\bigcap_{i=1}^{i=k}G_i = \phi$ 
(because 
$\displaystyle\bigcap_{i=1}^{i=k}G_i$ = $\displaystyle\bigcap_{i=1}^{i=k}(B_i - B)$ 
= $\displaystyle\bigcap_{i=1}^{i=k}B_i - B$ = $\phi$).
This intersection actually looks like 
$\displaystyle\bigcap_{i=1}^{i=k}(\displaystyle\bigcup_j\bar{C_j}) = \phi$. 
Taking complement, we have 
$\displaystyle\bigcup_{i=1}^{i=k}(\displaystyle\bigcap_j C_j) = S$. As a
result, we have that $S$ = $\displaystyle\bigcup_{i=i}^{i=k}P_i$, where $P_i$
represents the intersection of a finite number of sets from $\mathcal{C}$.
Because each of the sets in the intersection contains $P_i$, it suffices to
arbitrarily choose one of them for each $P_i$ to yield the desired set cover of
at most size $k$ for $S$. \hfill\fbox{}
