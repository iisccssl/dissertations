\chapter{Conclusions and Future Work}
\label{chapter:conclusions}

While it is ideal to proactively design software systems for security, economic
and practical considerations often preclude this in practice. This motivates the
need for retroactive techniques to analyze and transform legacy code for
security.

This dissertation has presented techniques to analyze and retofit legacy code
with mechanisms for authorization policy enforcement. It introduced
fingerprints, a low-level language to represent security-sensitive operations,
and showed that fingerprints can be used to identify locations in source code
that must be guarded by reference monitor calls. A central contribution of this
dissertation is a set of techniques based upon static and dynamic program
analysis to mine fingerprints by analyzing legacy code.

However, the work presented in this dissertation is not without its
limitations, and there are several directions in which to extend the work
presented in this dissertation to overcome these limitations.

\begin{enumerate}
%
\item \textbf{Reducing the size of the TCB.} As discussed in
\sectref{chapter:overview:tcb}, a key shortcoming of the approach presented in
this dissertation is that it increases the size of the TCB. In particular, the
legacy software system that is being retrofit is assumed to be benign. As a
result, the TCB must be extended to include the legacy software system. An
interesting future direction will be to examine techniques to enforce
authorization policies without increasing the size of the TCB. One way to
achieve this will be to redesign the operating system (which is traditionally
included in the TCB) to enforce application-level authorization policies. A key
challenge here will be to do so while remaining compatible with legacy
applications.

\item \textbf{Achieving soundness and completeness.} The fingerprint-based
approach presented in this dissertation is neither \textit{sound} nor
\textit{complete}.  Consequently, it is not completely automatic and requires
manual intervention to prune false positives (which result because the approach
is not sound) and identify false negatives (which result because the approach
is not complete).

The approach is not sound primarily because of the limited expressive power of
the language that is currently used to represent fingerprints. Two extensions
to the language will greatly improve its expressive power. First, the language
must be extended to express temporal relationships between code patterns,
\eg~using finite state automata to express fingerprints. Second, the language
must be extended to include data-flow information. Augmenting the language with
data-flow information will enable the expression of fine-grained information
that determines how resources are affected by a security-sensitive operation.

The approach lacks completeness for unsafe languages (such as C). As discussed
in \sectref{chapter:static:section:limitations}, this is a consequence of
type-safety violations, such as those caused by pointer aritmetic and direct
writes to memory. One approach to achieving completeness is to add extra
runtime checks (\textit{\'{a} la} CCured~\cite{nch+05,nmw02}) to the
retrofitted program to ensure type safety.  The resulting retrofitted system
will contain checks that enforce type safety in addition to those that enforce
site-specific authorization policies.
%
\item \textbf{Automating failure handling.} An important issue that has been
side-stepped by the approach presented in this dissertation is, ``how to handle
failed authorization policy checks?'' While the primary goal is to ensure that
a security-sensitive operation is never performed when an authorization policy
check fails, an important secondary goal is to ensure that the server notifies
clients in such a way that the failed check is handled gracefully by the
clients. For example, failure to create a new window or copy from a window must
not crash an \xclient\ that requested this operation. In current work, we defer
the task of implementing failure handling code to a human. An interesting
future direction will be to investigate automated techniques to gracefully
handle failure in a principled and automated way.
%
\item \textbf{Enforcing policies on unmodified binaries.} The approach
presented in this dissertation modifies legacy (source) code by retrofitting it
with authorization checks. An interesting future direction will be to
investigate techniques to enforce authorization policies on unmodified
binaries. Doing so will require constructing a runtime envionment that will
enforce authorization policies as code executes. Such a runtime environment
will enable the enforcement of authorization policies on commercial
off-the-shelf servers that may not be amenable to analysis and transformation.
% 
\end{enumerate}

The above directions provide fodder for much future research and
experimentation.
