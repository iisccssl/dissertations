Research in computer security has historically advocated Design for Security,
the principle that security must be proactively integrated into the design of a
system. While examples exist in the research literature of systems that have
been designed for security, there are few examples of such systems deployed in
the real world. Economic and practical considerations force developers to
abandon security and focus instead on functionality and performance, which are
more tangible than security. As a result, large bodies of legacy code often
have inadequate security mechanisms.  Security mechanisms are added to legacy
code on-demand using ad hoc and manual techniques, and the resulting systems
are often insecure.

This dissertation advocates the need for techniques to retrofit systems with
security mechanisms. In particular, it focuses on the problem of retrofitting
legacy code with mechanisms for authorization policy enforcement. It introduces
a new formalism, called fingerprints, to represent security-sensitive
operations. Fingerprints are code templates that represent accesses to
security-critical resources, and denote key steps needed to perform operations
on these resources. This dissertation develops both fingerprint mining and
fingerprint matching algorithms.

Fingerprint mining algorithms discover fingerprints of security-sensitive
operations by analyzing source code. This dissertation presents two novel
algorithms that use dynamic program analysis and static program analysis,
respectively, to mine fingerprints. The fingerprints so mined are used by the
fingerprint matching algorithm to statically locate security-sensitive
operations. Program transformation is then employed to statically modify source
code by adding authorization policy lookups at each location that performs a
security-sensitive operation.

The techniques developed in this dissertation have been applied to three
real-world systems. These case studies demonstrate that techniques based upon
program analysis and transformation offer a principled and automated
alternative to the ad hoc and manual techniques that are currently used to
retrofit legacy software with security mechanisms.
